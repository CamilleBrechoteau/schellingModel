% Options for packages loaded elsewhere
\PassOptionsToPackage{unicode}{hyperref}
\PassOptionsToPackage{hyphens}{url}
%
\documentclass[
]{article}
\usepackage{lmodern}
\usepackage{amssymb,amsmath}
\usepackage{ifxetex,ifluatex}
\ifnum 0\ifxetex 1\fi\ifluatex 1\fi=0 % if pdftex
  \usepackage[T1]{fontenc}
  \usepackage[utf8]{inputenc}
  \usepackage{textcomp} % provide euro and other symbols
\else % if luatex or xetex
  \usepackage{unicode-math}
  \defaultfontfeatures{Scale=MatchLowercase}
  \defaultfontfeatures[\rmfamily]{Ligatures=TeX,Scale=1}
\fi
% Use upquote if available, for straight quotes in verbatim environments
\IfFileExists{upquote.sty}{\usepackage{upquote}}{}
\IfFileExists{microtype.sty}{% use microtype if available
  \usepackage[]{microtype}
  \UseMicrotypeSet[protrusion]{basicmath} % disable protrusion for tt fonts
}{}
\makeatletter
\@ifundefined{KOMAClassName}{% if non-KOMA class
  \IfFileExists{parskip.sty}{%
    \usepackage{parskip}
  }{% else
    \setlength{\parindent}{0pt}
    \setlength{\parskip}{6pt plus 2pt minus 1pt}}
}{% if KOMA class
  \KOMAoptions{parskip=half}}
\makeatother
\usepackage{xcolor}
\IfFileExists{xurl.sty}{\usepackage{xurl}}{} % add URL line breaks if available
\IfFileExists{bookmark.sty}{\usepackage{bookmark}}{\usepackage{hyperref}}
\hypersetup{
  pdftitle={Projet1},
  pdfauthor={Camille BRECHOTEAU, Benjamin PERSON, Fanny MARTIN},
  hidelinks,
  pdfcreator={LaTeX via pandoc}}
\urlstyle{same} % disable monospaced font for URLs
\usepackage[margin=1in]{geometry}
\usepackage{color}
\usepackage{fancyvrb}
\newcommand{\VerbBar}{|}
\newcommand{\VERB}{\Verb[commandchars=\\\{\}]}
\DefineVerbatimEnvironment{Highlighting}{Verbatim}{commandchars=\\\{\}}
% Add ',fontsize=\small' for more characters per line
\usepackage{framed}
\definecolor{shadecolor}{RGB}{248,248,248}
\newenvironment{Shaded}{\begin{snugshade}}{\end{snugshade}}
\newcommand{\AlertTok}[1]{\textcolor[rgb]{0.94,0.16,0.16}{#1}}
\newcommand{\AnnotationTok}[1]{\textcolor[rgb]{0.56,0.35,0.01}{\textbf{\textit{#1}}}}
\newcommand{\AttributeTok}[1]{\textcolor[rgb]{0.77,0.63,0.00}{#1}}
\newcommand{\BaseNTok}[1]{\textcolor[rgb]{0.00,0.00,0.81}{#1}}
\newcommand{\BuiltInTok}[1]{#1}
\newcommand{\CharTok}[1]{\textcolor[rgb]{0.31,0.60,0.02}{#1}}
\newcommand{\CommentTok}[1]{\textcolor[rgb]{0.56,0.35,0.01}{\textit{#1}}}
\newcommand{\CommentVarTok}[1]{\textcolor[rgb]{0.56,0.35,0.01}{\textbf{\textit{#1}}}}
\newcommand{\ConstantTok}[1]{\textcolor[rgb]{0.00,0.00,0.00}{#1}}
\newcommand{\ControlFlowTok}[1]{\textcolor[rgb]{0.13,0.29,0.53}{\textbf{#1}}}
\newcommand{\DataTypeTok}[1]{\textcolor[rgb]{0.13,0.29,0.53}{#1}}
\newcommand{\DecValTok}[1]{\textcolor[rgb]{0.00,0.00,0.81}{#1}}
\newcommand{\DocumentationTok}[1]{\textcolor[rgb]{0.56,0.35,0.01}{\textbf{\textit{#1}}}}
\newcommand{\ErrorTok}[1]{\textcolor[rgb]{0.64,0.00,0.00}{\textbf{#1}}}
\newcommand{\ExtensionTok}[1]{#1}
\newcommand{\FloatTok}[1]{\textcolor[rgb]{0.00,0.00,0.81}{#1}}
\newcommand{\FunctionTok}[1]{\textcolor[rgb]{0.00,0.00,0.00}{#1}}
\newcommand{\ImportTok}[1]{#1}
\newcommand{\InformationTok}[1]{\textcolor[rgb]{0.56,0.35,0.01}{\textbf{\textit{#1}}}}
\newcommand{\KeywordTok}[1]{\textcolor[rgb]{0.13,0.29,0.53}{\textbf{#1}}}
\newcommand{\NormalTok}[1]{#1}
\newcommand{\OperatorTok}[1]{\textcolor[rgb]{0.81,0.36,0.00}{\textbf{#1}}}
\newcommand{\OtherTok}[1]{\textcolor[rgb]{0.56,0.35,0.01}{#1}}
\newcommand{\PreprocessorTok}[1]{\textcolor[rgb]{0.56,0.35,0.01}{\textit{#1}}}
\newcommand{\RegionMarkerTok}[1]{#1}
\newcommand{\SpecialCharTok}[1]{\textcolor[rgb]{0.00,0.00,0.00}{#1}}
\newcommand{\SpecialStringTok}[1]{\textcolor[rgb]{0.31,0.60,0.02}{#1}}
\newcommand{\StringTok}[1]{\textcolor[rgb]{0.31,0.60,0.02}{#1}}
\newcommand{\VariableTok}[1]{\textcolor[rgb]{0.00,0.00,0.00}{#1}}
\newcommand{\VerbatimStringTok}[1]{\textcolor[rgb]{0.31,0.60,0.02}{#1}}
\newcommand{\WarningTok}[1]{\textcolor[rgb]{0.56,0.35,0.01}{\textbf{\textit{#1}}}}
\usepackage{graphicx,grffile}
\makeatletter
\def\maxwidth{\ifdim\Gin@nat@width>\linewidth\linewidth\else\Gin@nat@width\fi}
\def\maxheight{\ifdim\Gin@nat@height>\textheight\textheight\else\Gin@nat@height\fi}
\makeatother
% Scale images if necessary, so that they will not overflow the page
% margins by default, and it is still possible to overwrite the defaults
% using explicit options in \includegraphics[width, height, ...]{}
\setkeys{Gin}{width=\maxwidth,height=\maxheight,keepaspectratio}
% Set default figure placement to htbp
\makeatletter
\def\fps@figure{htbp}
\makeatother
\setlength{\emergencystretch}{3em} % prevent overfull lines
\providecommand{\tightlist}{%
  \setlength{\itemsep}{0pt}\setlength{\parskip}{0pt}}
\setcounter{secnumdepth}{-\maxdimen} % remove section numbering

\title{Projet1}
\author{Camille BRECHOTEAU, Benjamin PERSON, Fanny MARTIN}
\date{12/12/2020}

\begin{document}
\maketitle

\hypertarget{i--introduction}{%
\subsection{I- INTRODUCTION}\label{i--introduction}}

\begin{verbatim}
      Dans le cadre de notre troisième année du cycle de Licence en biologie-informatique ou en biologie option bio-informatique (BI) à l’Université Evry Val-d ’Essonne, il nous est proposé un projet de groupe nous permettant de mettre en pratique nos connaissances en programmation R au travers d’un projet ayant pour finalité la reproduction du modèle de ségrégation spatiale proposée dans les années 1970 par Thomas C. Schelling  qui a marqué les esprits en raison de l’effet pervers qu’elle suggérait : il existerait un lien entre une ségrégation spatiale et le résultat collectif de décisions individuelles qui ne visent pas à une telle ségrégation. Ainsi que l’étude de la densité de population et que du seuil d’intolérance dans ce dernier.
\end{verbatim}

\hypertarget{ii-fonction-simu}{%
\subsection{II-FONCTION SIMU}\label{ii-fonction-simu}}

\hypertarget{a-one.simu-le-moduxe8le-de-schelling}{%
\section{A-One.simu, le modèle de
Schelling}\label{a-one.simu-le-moduxe8le-de-schelling}}

On créer une fonction nommée ``one.simu'' qui permettra de réaliser les
dynamiques des populations bleues et rougse selon le modèle de
Schelling. Elle prend en paramètre un seuil (initaliser à 3), une taille
de coté (longueur = larguer, initialisée à 50) ainsi que u (initialiser
à 1 qui permet de faire varier la densité d'espaces vacants). On va
créer une matrice carrée taille N*N avec trois différentes valeurs
possibles, 0 pour les espaces vacants (dont la proportion changera selon
la valeur de u), 1 pour les agents Bleus et 2 pour les agents Rouges.

On a utilisé une boucle while et 2 condition d'arrêt, une qui donne un
nombre d'itération maximal (on a choisi 1000) et l'autre qui stopera la
boucle en cas d'impossibilité de continuer.

A chaque itération : On commence par déterminer le nombre de voisins
Bleus et le nombre de voisins Rouges pour toutes les cases de la
matrice.

On détermine ensuite les listes des agents insatisfaits (bleu,export1
puis rouge, export2) et les listes des endroits vacant qui satisferai
les bleu et rouges (import1 et import2).

On vérifie la condition d'arrêt, ``stop'', si jamais l'une des 4 listes
determinée précedement est vide il faut stopper la boucle while.

Si la boucle peut continuer, on va sélectionner une position au hasard
parmi les agents Bleus/Rouges insatisfaits ainsi qu'une case vacante
parmi ``import1'' et ``import2''

Enfin; on placera le Bleu insatisfait dans l'espace vacant qui lui
permet d'être satisfait et on fera de même pour le Rouge. On
implémentera i et la boucle recommencera.

Une fois la boucle terminé on renvoie la matrice modifiée.

\begin{Shaded}
\begin{Highlighting}[]
\CommentTok{#APPLIQUE LE MODE DE Schelling}
\NormalTok{one.simu<-}\ControlFlowTok{function}\NormalTok{(}\DataTypeTok{seuil=}\DecValTok{3}\NormalTok{,}\DataTypeTok{N=}\DecValTok{50}\NormalTok{,}\DataTypeTok{u=}\DecValTok{1}\NormalTok{)\{}
\NormalTok{M <-}\StringTok{ }\KeywordTok{matrix}\NormalTok{(}\KeywordTok{sample}\NormalTok{(}\KeywordTok{c}\NormalTok{(}\DecValTok{0}\NormalTok{,}\DecValTok{1}\NormalTok{,}\DecValTok{2}\NormalTok{), }\DataTypeTok{replace =} \OtherTok{TRUE}\NormalTok{, }\DataTypeTok{prob =} \KeywordTok{c}\NormalTok{(u,}\DecValTok{1}\NormalTok{,}\DecValTok{1}\NormalTok{), }\DataTypeTok{size =}\NormalTok{ N}\OperatorTok{^}\DecValTok{2}\NormalTok{), }\DataTypeTok{nrow =}\NormalTok{ N)}

\NormalTok{V<-}\StringTok{ }\DecValTok{1}\OperatorTok{:}\NormalTok{N}
\NormalTok{G <-}\StringTok{ }\KeywordTok{c}\NormalTok{(V[}\OperatorTok{-}\DecValTok{1}\NormalTok{],V[}\DecValTok{1}\NormalTok{])}
\NormalTok{D <-}\StringTok{ }\KeywordTok{c}\NormalTok{(V[N],V[}\OperatorTok{-}\NormalTok{N])}

\NormalTok{i<-}\DecValTok{1}
\NormalTok{stop<-}\OtherTok{FALSE}

\ControlFlowTok{while}\NormalTok{(i}\OperatorTok{<=}\DecValTok{1000} \OperatorTok{&}\StringTok{ }\NormalTok{stop}\OperatorTok{==}\OtherTok{FALSE}\NormalTok{)}
\NormalTok{\{}

\NormalTok{nb1 <-}\StringTok{ }\NormalTok{(M[D,] }\OperatorTok{==}\StringTok{ }\DecValTok{1}\NormalTok{) }\OperatorTok{+}\StringTok{ }\NormalTok{(M[G,] }\OperatorTok{==}\StringTok{ }\DecValTok{1}\NormalTok{) }\OperatorTok{+}\StringTok{ }\NormalTok{(M[,D] }\OperatorTok{==}\StringTok{ }\DecValTok{1}\NormalTok{) }\OperatorTok{+}\StringTok{ }\NormalTok{(M[,G] }\OperatorTok{==}\StringTok{ }\DecValTok{1}\NormalTok{) }\OperatorTok{+}\StringTok{ }\NormalTok{(M[D,D] }\OperatorTok{==}\StringTok{ }\DecValTok{1}\NormalTok{) }\OperatorTok{+}\StringTok{ }\NormalTok{(M[G,D] }\OperatorTok{==}\StringTok{ }\DecValTok{1}\NormalTok{) }\OperatorTok{+}\StringTok{ }\NormalTok{(M [D,G] }\OperatorTok{==}\StringTok{ }\DecValTok{1}\NormalTok{) }\OperatorTok{+}\StringTok{ }\NormalTok{(M[G,G] }\OperatorTok{==}\StringTok{ }\DecValTok{1}\NormalTok{)}
\NormalTok{nb2 <-}\StringTok{ }\NormalTok{(M[D,] }\OperatorTok{==}\StringTok{ }\DecValTok{2}\NormalTok{) }\OperatorTok{+}\StringTok{ }\NormalTok{(M[G,] }\OperatorTok{==}\StringTok{ }\DecValTok{2}\NormalTok{) }\OperatorTok{+}\StringTok{ }\NormalTok{(M[,D] }\OperatorTok{==}\StringTok{ }\DecValTok{2}\NormalTok{) }\OperatorTok{+}\StringTok{ }\NormalTok{(M[,G] }\OperatorTok{==}\StringTok{ }\DecValTok{2}\NormalTok{) }\OperatorTok{+}\StringTok{ }\NormalTok{(M[D,D] }\OperatorTok{==}\StringTok{ }\DecValTok{2}\NormalTok{) }\OperatorTok{+}\StringTok{ }\NormalTok{(M[G,D] }\OperatorTok{==}\StringTok{ }\DecValTok{2}\NormalTok{) }\OperatorTok{+}\StringTok{ }\NormalTok{(M [D,G] }\OperatorTok{==}\StringTok{ }\DecValTok{2}\NormalTok{) }\OperatorTok{+}\StringTok{ }\NormalTok{(M[G,G] }\OperatorTok{==}\StringTok{ }\DecValTok{2}\NormalTok{)}
\CommentTok{#nb1, nb2 = nombre de 1 et de 2 autour de chaque case}

\CommentTok{#A DEPLACER}
\NormalTok{export1 <-}\StringTok{ }\NormalTok{(nb1 }\OperatorTok{<}\StringTok{ }\NormalTok{seuil) }\OperatorTok{&}\StringTok{ }\NormalTok{(M }\OperatorTok{==}\StringTok{ }\DecValTok{1}\NormalTok{)}
\NormalTok{export2 <-}\StringTok{ }\NormalTok{(nb2 }\OperatorTok{<}\StringTok{ }\NormalTok{seuil) }\OperatorTok{&}\StringTok{ }\NormalTok{(M }\OperatorTok{==}\StringTok{ }\DecValTok{2}\NormalTok{)}
\NormalTok{export1}
\NormalTok{export2}

\CommentTok{#OU DEPLACER}
\NormalTok{import1 <-}\StringTok{ }\NormalTok{(nb1 }\OperatorTok{>=}\StringTok{ }\NormalTok{seuil) }\OperatorTok{&}\StringTok{ }\NormalTok{(M }\OperatorTok{==}\StringTok{ }\DecValTok{0}\NormalTok{)}
\NormalTok{import2 <-}\StringTok{ }\NormalTok{(nb2 }\OperatorTok{>=}\StringTok{ }\NormalTok{seuil) }\OperatorTok{&}\StringTok{ }\NormalTok{(M }\OperatorTok{==}\StringTok{ }\DecValTok{0}\NormalTok{)}
\NormalTok{import1}
\NormalTok{import2}


\CommentTok{#CHOISIR UN AGENT INSATISFAIT}
\KeywordTok{which}\NormalTok{(export1)}
\KeywordTok{which}\NormalTok{(export2)}
\ControlFlowTok{if}\NormalTok{( }\KeywordTok{length}\NormalTok{(}\KeywordTok{which}\NormalTok{(export1))}\OperatorTok{==}\DecValTok{0} \OperatorTok{|}\StringTok{ }\KeywordTok{length}\NormalTok{(}\KeywordTok{which}\NormalTok{(export2))}\OperatorTok{==}\DecValTok{0} \OperatorTok{|}\StringTok{ }\KeywordTok{length}\NormalTok{(}\KeywordTok{which}\NormalTok{(import1))}\OperatorTok{==}\DecValTok{0} \OperatorTok{|}\StringTok{ }\KeywordTok{length}\NormalTok{(}\KeywordTok{which}\NormalTok{(import2))}\OperatorTok{==}\DecValTok{0}\NormalTok{)\{}
\NormalTok{  stop<-}\OtherTok{TRUE}
\NormalTok{\}}
  
\ControlFlowTok{if}\NormalTok{(stop}\OperatorTok{==}\OtherTok{FALSE}\NormalTok{)\{}

\NormalTok{agent1<-}\KeywordTok{sample}\NormalTok{(}\DataTypeTok{x=}\KeywordTok{which}\NormalTok{(export1),}\DataTypeTok{size=}\DecValTok{1}\NormalTok{)}
\NormalTok{agent2<-}\KeywordTok{sample}\NormalTok{(}\DataTypeTok{x=}\KeywordTok{which}\NormalTok{(export2),}\DataTypeTok{size=}\DecValTok{1}\NormalTok{)}
\NormalTok{agent1}
\NormalTok{agent2}

\CommentTok{#CHOSIR ESPACE VACANT SATISFAISANT}
\CommentTok{#v1<-which(import1)}
\CommentTok{#v2<-which(import2)}

\NormalTok{vacant1<-}\KeywordTok{sample}\NormalTok{(}\DataTypeTok{x=}\KeywordTok{which}\NormalTok{(import1),}\DataTypeTok{size=}\DecValTok{1}\NormalTok{)}
\NormalTok{vacant2<-}\KeywordTok{sample}\NormalTok{(}\DataTypeTok{x=}\KeywordTok{which}\NormalTok{(import2),}\DataTypeTok{size=}\DecValTok{1}\NormalTok{)}
\NormalTok{vacant1}
\NormalTok{vacant2}

\CommentTok{#DEMENAGEMENT }
\NormalTok{M[agent1]<-}\DecValTok{0}
\NormalTok{M[vacant1]<-}\DecValTok{1}
\NormalTok{M[agent2]<-}\DecValTok{0}
\NormalTok{M[vacant2]<-}\DecValTok{2}
\NormalTok{\}}
\NormalTok{i<-i}\OperatorTok{+}\DecValTok{1}
\NormalTok{\}}
\NormalTok{M}
\KeywordTok{return}\NormalTok{(M)}
\NormalTok{\}}
\end{Highlighting}
\end{Shaded}

\hypertarget{b-two.simu-meusre-dagruxe9gation}{%
\section{B-Two.simu, meusre
d'agrégation}\label{b-two.simu-meusre-dagruxe9gation}}

On va par la suite mesurer l'agrégation. On va créer une autre fonction
qui prendra en paramètre une matrice M.Il existe plusieurs méthodes, on
a choisi de calculer le N0 qui correspond au nombre d'agents entouré de
8 agents de même type que lui (ex: un Bleu entouré de 8 Bleus).Pour
cela, on recalcule ``nb1'' et ``nb2''.

On utilise un accumulateur ``N0'' que l'on implémente quand on rencontre
un agent entouré par 8 agents du même type en parcourant la matrice
(grâce à la boucle for). La fonction retourne la valeur N0.

\begin{Shaded}
\begin{Highlighting}[]
\CommentTok{#MESURE AGREAGTION N0}
\NormalTok{two.simu<-}\ControlFlowTok{function}\NormalTok{(M,}\DataTypeTok{N=}\DecValTok{50}\NormalTok{)\{}
 
\NormalTok{V<-}\StringTok{ }\DecValTok{1}\OperatorTok{:}\NormalTok{N}
\NormalTok{G <-}\StringTok{ }\KeywordTok{c}\NormalTok{(V[}\OperatorTok{-}\DecValTok{1}\NormalTok{],V[}\DecValTok{1}\NormalTok{])}
\NormalTok{D <-}\StringTok{ }\KeywordTok{c}\NormalTok{(V[N],V[}\OperatorTok{-}\NormalTok{N])}
  
  
\NormalTok{nb1 <-}\StringTok{ }\NormalTok{(M[D,] }\OperatorTok{==}\StringTok{ }\DecValTok{1}\NormalTok{) }\OperatorTok{+}\StringTok{ }\NormalTok{(M[G,] }\OperatorTok{==}\StringTok{ }\DecValTok{1}\NormalTok{) }\OperatorTok{+}\StringTok{ }\NormalTok{(M[,D] }\OperatorTok{==}\StringTok{ }\DecValTok{1}\NormalTok{) }\OperatorTok{+}\StringTok{ }\NormalTok{(M[,G] }\OperatorTok{==}\StringTok{ }\DecValTok{1}\NormalTok{) }\OperatorTok{+}\StringTok{ }\NormalTok{(M[D,D] }\OperatorTok{==}\StringTok{ }\DecValTok{1}\NormalTok{) }\OperatorTok{+}\StringTok{ }\NormalTok{(M[G,D] }\OperatorTok{==}\StringTok{ }\DecValTok{1}\NormalTok{) }\OperatorTok{+}\StringTok{ }\NormalTok{(M [D,G] }\OperatorTok{==}\StringTok{ }\DecValTok{1}\NormalTok{) }\OperatorTok{+}\StringTok{ }\NormalTok{(M[G,G] }\OperatorTok{==}\StringTok{ }\DecValTok{1}\NormalTok{)}
\NormalTok{nb2 <-}\StringTok{ }\NormalTok{(M[D,] }\OperatorTok{==}\StringTok{ }\DecValTok{2}\NormalTok{) }\OperatorTok{+}\StringTok{ }\NormalTok{(M[G,] }\OperatorTok{==}\StringTok{ }\DecValTok{2}\NormalTok{) }\OperatorTok{+}\StringTok{ }\NormalTok{(M[,D] }\OperatorTok{==}\StringTok{ }\DecValTok{2}\NormalTok{) }\OperatorTok{+}\StringTok{ }\NormalTok{(M[,G] }\OperatorTok{==}\StringTok{ }\DecValTok{2}\NormalTok{) }\OperatorTok{+}\StringTok{ }\NormalTok{(M[D,D] }\OperatorTok{==}\StringTok{ }\DecValTok{2}\NormalTok{) }\OperatorTok{+}\StringTok{ }\NormalTok{(M[G,D] }\OperatorTok{==}\StringTok{ }\DecValTok{2}\NormalTok{) }\OperatorTok{+}\StringTok{ }\NormalTok{(M [D,G] }\OperatorTok{==}\StringTok{ }\DecValTok{2}\NormalTok{) }\OperatorTok{+}\StringTok{ }\NormalTok{(M[G,G] }\OperatorTok{==}\StringTok{ }\DecValTok{2}\NormalTok{)}
\CommentTok{#nb1, nb2 = nombre de 1 et de 2 autour de chaque case}
\NormalTok{nb1}
\NormalTok{nb2}


\NormalTok{N0<-}\DecValTok{0}
\CommentTok{#parcour de la matrice, }
\ControlFlowTok{for}\NormalTok{(j }\ControlFlowTok{in} \DecValTok{1}\OperatorTok{:}\NormalTok{(N}\OperatorTok{*}\NormalTok{N))\{}
  
  \ControlFlowTok{if}\NormalTok{ (M[j]}\OperatorTok{==}\DecValTok{1} \OperatorTok{&}\StringTok{ }\NormalTok{nb1[j]}\OperatorTok{==}\DecValTok{8}\NormalTok{)\{}
\NormalTok{    N0<-N0}\OperatorTok{+}\DecValTok{1}
    
\NormalTok{  \}}
  \ControlFlowTok{if}\NormalTok{ (M[j]}\OperatorTok{==}\DecValTok{2} \OperatorTok{&}\StringTok{ }\NormalTok{nb2[j]}\OperatorTok{==}\DecValTok{8}\NormalTok{ )\{}
\NormalTok{    N0<-N0}\OperatorTok{+}\DecValTok{1}
\NormalTok{  \}}

\NormalTok{\}}


\KeywordTok{return}\NormalTok{ (N0)}
\NormalTok{\}}
\end{Highlighting}
\end{Shaded}

\hypertarget{iii--simulation}{%
\subsection{III- SIMULATION}\label{iii--simulation}}

On va maintenant utiliser les 2 fonctions afin d'observer et analyser
les déplacements de populations en faisant varier 2 paramètres : le
seuil d'intolérance et la densité d'espaces vacants.

On va utiliser 3 seuils différents : 3,4 et 5. Au-delà de 5 les agents
sont trop souvent insatisfait et en dessous de 3 quasiment toujours
satisfait (selon l'article) Pour ce qui est de la densita, on va varier
entre 2, 15 et 33\% d'espaces vacants.

\hypertarget{a-ruxe9sultats}{%
\section{A-Résultats}\label{a-ruxe9sultats}}

Voici des matrices avec les différents seuils et densités :

Seuil: 3, Densité: 2\%

\begin{Shaded}
\begin{Highlighting}[]
\NormalTok{s3d02EX<-}\KeywordTok{one.simu}\NormalTok{(}\DecValTok{3}\NormalTok{,}\DecValTok{50}\NormalTok{,}\FloatTok{0.04}\NormalTok{)}\CommentTok{#densité 2%}
\KeywordTok{image}\NormalTok{(s3d02EX,}\DataTypeTok{col=}\KeywordTok{c}\NormalTok{(}\StringTok{"gray99"}\NormalTok{,}\StringTok{"darkblue"}\NormalTok{,}\StringTok{"darkred"}\NormalTok{))}
\end{Highlighting}
\end{Shaded}

\includegraphics{projet1_files/figure-latex/unnamed-chunk-3-1.pdf}

Seuil: 3, Densité: 15\%

\begin{Shaded}
\begin{Highlighting}[]
\NormalTok{s3d15EX<-}\KeywordTok{one.simu}\NormalTok{(}\DecValTok{3}\NormalTok{,}\DecValTok{50}\NormalTok{,}\FloatTok{0.35}\NormalTok{)}\CommentTok{#densité 15%}
\KeywordTok{image}\NormalTok{(s3d15EX,}\DataTypeTok{col=}\KeywordTok{c}\NormalTok{(}\StringTok{"gray99"}\NormalTok{,}\StringTok{"darkblue"}\NormalTok{,}\StringTok{"darkred"}\NormalTok{))}
\end{Highlighting}
\end{Shaded}

\includegraphics{projet1_files/figure-latex/unnamed-chunk-4-1.pdf}

Seuil: 3, Densité: 33\%

\begin{Shaded}
\begin{Highlighting}[]
\NormalTok{s3d33EX<-}\KeywordTok{one.simu}\NormalTok{(}\DecValTok{3}\NormalTok{,}\DecValTok{50}\NormalTok{,}\DecValTok{1}\NormalTok{)}\CommentTok{#densité 33%}
\KeywordTok{image}\NormalTok{(s3d33EX,}\DataTypeTok{col=}\KeywordTok{c}\NormalTok{(}\StringTok{"gray99"}\NormalTok{,}\StringTok{"darkblue"}\NormalTok{,}\StringTok{"darkred"}\NormalTok{))}
\end{Highlighting}
\end{Shaded}

\includegraphics{projet1_files/figure-latex/unnamed-chunk-5-1.pdf}

Seuil: 4, Densité 2\%

\begin{Shaded}
\begin{Highlighting}[]
\NormalTok{s4d02EX<-}\KeywordTok{one.simu}\NormalTok{(}\DecValTok{4}\NormalTok{,}\DecValTok{50}\NormalTok{,}\FloatTok{0.04}\NormalTok{)}\CommentTok{#densité 2%}
\KeywordTok{image}\NormalTok{(s4d02EX,}\DataTypeTok{col=}\KeywordTok{c}\NormalTok{(}\StringTok{"gray99"}\NormalTok{,}\StringTok{"darkblue"}\NormalTok{,}\StringTok{"darkred"}\NormalTok{))}
\end{Highlighting}
\end{Shaded}

\includegraphics{projet1_files/figure-latex/unnamed-chunk-6-1.pdf}

Seuil: 4, Densité 15\%

\begin{Shaded}
\begin{Highlighting}[]
\NormalTok{s4d15EX<-}\KeywordTok{one.simu}\NormalTok{(}\DecValTok{4}\NormalTok{,}\DecValTok{50}\NormalTok{,}\FloatTok{0.35}\NormalTok{)}\CommentTok{#densité 15%}
\KeywordTok{image}\NormalTok{(s4d15EX,}\DataTypeTok{col=}\KeywordTok{c}\NormalTok{(}\StringTok{"gray99"}\NormalTok{,}\StringTok{"darkblue"}\NormalTok{,}\StringTok{"darkred"}\NormalTok{))}
\end{Highlighting}
\end{Shaded}

\includegraphics{projet1_files/figure-latex/unnamed-chunk-7-1.pdf}

Seuil: 4, Densité 33\%

\begin{Shaded}
\begin{Highlighting}[]
\NormalTok{s4d33EX<-}\KeywordTok{one.simu}\NormalTok{(}\DecValTok{4}\NormalTok{,}\DecValTok{50}\NormalTok{,}\DecValTok{1}\NormalTok{)}\CommentTok{#densité 33%}
\KeywordTok{image}\NormalTok{(s4d33EX,}\DataTypeTok{col=}\KeywordTok{c}\NormalTok{(}\StringTok{"gray99"}\NormalTok{,}\StringTok{"darkblue"}\NormalTok{,}\StringTok{"darkred"}\NormalTok{))}
\end{Highlighting}
\end{Shaded}

\includegraphics{projet1_files/figure-latex/unnamed-chunk-8-1.pdf}

Seuil: 5, Densité 2\%

\begin{Shaded}
\begin{Highlighting}[]
\NormalTok{s5d02EX<-}\KeywordTok{one.simu}\NormalTok{(}\DecValTok{5}\NormalTok{,}\DecValTok{50}\NormalTok{,}\FloatTok{0.04}\NormalTok{)}\CommentTok{#densité 2%}
\KeywordTok{image}\NormalTok{(s5d02EX,}\DataTypeTok{col=}\KeywordTok{c}\NormalTok{(}\StringTok{"gray99"}\NormalTok{,}\StringTok{"darkblue"}\NormalTok{,}\StringTok{"darkred"}\NormalTok{))}
\end{Highlighting}
\end{Shaded}

\includegraphics{projet1_files/figure-latex/unnamed-chunk-9-1.pdf}

Seuil: 5, Densité 15\%

\begin{Shaded}
\begin{Highlighting}[]
\NormalTok{s5d15EX<-}\KeywordTok{one.simu}\NormalTok{(}\DecValTok{5}\NormalTok{,}\DecValTok{50}\NormalTok{,}\FloatTok{0.35}\NormalTok{)}\CommentTok{#densité 15%}
\KeywordTok{image}\NormalTok{(s5d15EX,}\DataTypeTok{col=}\KeywordTok{c}\NormalTok{(}\StringTok{"gray99"}\NormalTok{,}\StringTok{"darkblue"}\NormalTok{,}\StringTok{"darkred"}\NormalTok{))}
\end{Highlighting}
\end{Shaded}

\includegraphics{projet1_files/figure-latex/unnamed-chunk-10-1.pdf}

Seuil: 5, Densité 2\%

\begin{Shaded}
\begin{Highlighting}[]
\NormalTok{s5d33EX<-}\KeywordTok{one.simu}\NormalTok{(}\DecValTok{5}\NormalTok{,}\DecValTok{50}\NormalTok{,}\DecValTok{1}\NormalTok{)}\CommentTok{#densité 33%}
\KeywordTok{image}\NormalTok{(s5d33EX,}\DataTypeTok{col=}\KeywordTok{c}\NormalTok{(}\StringTok{"gray99"}\NormalTok{,}\StringTok{"darkblue"}\NormalTok{,}\StringTok{"darkred"}\NormalTok{))}
\end{Highlighting}
\end{Shaded}

\includegraphics{projet1_files/figure-latex/unnamed-chunk-11-1.pdf} Afin
pouvoir analyser l'effet du seuil et de la densité, on va recourir à
``replicate''. On va exécuter 100x chaque combinaison de seuil et
densité. On obtiendra des listes de N0 de tous les seuils et densités.

\begin{Shaded}
\begin{Highlighting}[]
\CommentTok{#simu avec seuil 3 et densité de 20 a 60%}
\NormalTok{s3d02<-}\KeywordTok{replicate}\NormalTok{(}\DecValTok{100}\NormalTok{,}\KeywordTok{two.simu}\NormalTok{(}\KeywordTok{one.simu}\NormalTok{(}\DecValTok{3}\NormalTok{,}\DecValTok{50}\NormalTok{,}\FloatTok{0.04}\NormalTok{)))}
\NormalTok{s3d15<-}\KeywordTok{replicate}\NormalTok{(}\DecValTok{100}\NormalTok{,}\KeywordTok{two.simu}\NormalTok{(}\KeywordTok{one.simu}\NormalTok{(}\DecValTok{3}\NormalTok{,}\DecValTok{50}\NormalTok{,}\FloatTok{0.35}\NormalTok{)))}
\NormalTok{s3d33<-}\KeywordTok{replicate}\NormalTok{(}\DecValTok{100}\NormalTok{,}\KeywordTok{two.simu}\NormalTok{(}\KeywordTok{one.simu}\NormalTok{(}\DecValTok{3}\NormalTok{,}\DecValTok{50}\NormalTok{,}\DecValTok{1}\NormalTok{)))}

\CommentTok{#simu avec seuil 4 et densité de 20 a 60%}
\NormalTok{s4d02<-}\KeywordTok{replicate}\NormalTok{(}\DecValTok{100}\NormalTok{,}\KeywordTok{two.simu}\NormalTok{(}\KeywordTok{one.simu}\NormalTok{(}\DecValTok{4}\NormalTok{,}\DecValTok{50}\NormalTok{,}\FloatTok{0.04}\NormalTok{)))}
\NormalTok{s4d15<-}\KeywordTok{replicate}\NormalTok{(}\DecValTok{100}\NormalTok{,}\KeywordTok{two.simu}\NormalTok{(}\KeywordTok{one.simu}\NormalTok{(}\DecValTok{4}\NormalTok{,}\DecValTok{50}\NormalTok{,}\FloatTok{0.35}\NormalTok{)))}
\NormalTok{s4d33<-}\KeywordTok{replicate}\NormalTok{(}\DecValTok{100}\NormalTok{,}\KeywordTok{two.simu}\NormalTok{(}\KeywordTok{one.simu}\NormalTok{(}\DecValTok{4}\NormalTok{,}\DecValTok{50}\NormalTok{,}\DecValTok{1}\NormalTok{)))}


\CommentTok{#simu avec seul 5 et densité de 20 a 60%}
\NormalTok{s5d02<-}\KeywordTok{replicate}\NormalTok{(}\DecValTok{100}\NormalTok{,}\KeywordTok{two.simu}\NormalTok{(}\KeywordTok{one.simu}\NormalTok{(}\DecValTok{5}\NormalTok{,}\DecValTok{50}\NormalTok{,}\FloatTok{0.04}\NormalTok{)))}
\NormalTok{s5d15<-}\KeywordTok{replicate}\NormalTok{(}\DecValTok{100}\NormalTok{,}\KeywordTok{two.simu}\NormalTok{(}\KeywordTok{one.simu}\NormalTok{(}\DecValTok{5}\NormalTok{,}\DecValTok{50}\NormalTok{,}\FloatTok{0.35}\NormalTok{)))}
\NormalTok{s5d33<-}\KeywordTok{replicate}\NormalTok{(}\DecValTok{100}\NormalTok{,}\KeywordTok{two.simu}\NormalTok{(}\KeywordTok{one.simu}\NormalTok{(}\DecValTok{5}\NormalTok{,}\DecValTok{50}\NormalTok{,}\DecValTok{1}\NormalTok{)))}
\end{Highlighting}
\end{Shaded}

On va créer un data.frame. On va utiliser les moyennes des mesures
d'agrégation obtenues (N0).

\begin{Shaded}
\begin{Highlighting}[]
\NormalTok{d<-}\KeywordTok{c}\NormalTok{(}\DecValTok{2}\NormalTok{,}\DecValTok{15}\NormalTok{,}\DecValTok{33}\NormalTok{)}
\NormalTok{echelle<-}\KeywordTok{c}\NormalTok{(}\DecValTok{0}\NormalTok{,}\DecValTok{750}\NormalTok{,}\DecValTok{1500}\NormalTok{)}
\NormalTok{x3<-}\KeywordTok{c}\NormalTok{(}\KeywordTok{mean}\NormalTok{(s3d02),}\KeywordTok{mean}\NormalTok{(s3d15),}\KeywordTok{mean}\NormalTok{(s3d33))}
\NormalTok{x4<-}\KeywordTok{c}\NormalTok{(}\KeywordTok{mean}\NormalTok{(s4d02),}\KeywordTok{mean}\NormalTok{(s4d15),}\KeywordTok{mean}\NormalTok{(s4d33)) }
\NormalTok{x5<-}\KeywordTok{c}\NormalTok{(}\KeywordTok{mean}\NormalTok{(s5d02),}\KeywordTok{mean}\NormalTok{(s5d15),}\KeywordTok{mean}\NormalTok{(s5d33))}

\NormalTok{d}
\end{Highlighting}
\end{Shaded}

\begin{verbatim}
## [1]  2 15 33
\end{verbatim}

\begin{Shaded}
\begin{Highlighting}[]
\NormalTok{x3}
\end{Highlighting}
\end{Shaded}

\begin{verbatim}
## [1] 620.42 234.82 130.06
\end{verbatim}

\begin{Shaded}
\begin{Highlighting}[]
\NormalTok{x4}
\end{Highlighting}
\end{Shaded}

\begin{verbatim}
## [1] 1308.65 1039.02  809.17
\end{verbatim}

\begin{Shaded}
\begin{Highlighting}[]
\NormalTok{x5}
\end{Highlighting}
\end{Shaded}

\begin{verbatim}
## [1]  74.85 101.38  60.31
\end{verbatim}

\begin{Shaded}
\begin{Highlighting}[]
\NormalTok{liste<-}\KeywordTok{list}\NormalTok{(d,x3,x4,x5,echelle)}
\NormalTok{donnees<-}\KeywordTok{data.frame}\NormalTok{(liste)}
\KeywordTok{colnames}\NormalTok{(donnees)<-}\KeywordTok{c}\NormalTok{(}\StringTok{"densité"}\NormalTok{,}\StringTok{"mesure agrégation seuil 3"}\NormalTok{, }\StringTok{"mesure agrégation seuil 4"}\NormalTok{,}\StringTok{"mesure agrégation seuil 5"}\NormalTok{,}\StringTok{"echelle"}\NormalTok{)}
\KeywordTok{print}\NormalTok{(}\KeywordTok{summary}\NormalTok{(donnees))}
\end{Highlighting}
\end{Shaded}

\begin{verbatim}
##     densité      mesure agrégation seuil 3 mesure agrégation seuil 4
##  Min.   : 2.00   Min.   :130.1             Min.   : 809.2           
##  1st Qu.: 8.50   1st Qu.:182.4             1st Qu.: 924.1           
##  Median :15.00   Median :234.8             Median :1039.0           
##  Mean   :16.67   Mean   :328.4             Mean   :1052.3           
##  3rd Qu.:24.00   3rd Qu.:427.6             3rd Qu.:1173.8           
##  Max.   :33.00   Max.   :620.4             Max.   :1308.7           
##  mesure agrégation seuil 5    echelle    
##  Min.   : 60.31            Min.   :   0  
##  1st Qu.: 67.58            1st Qu.: 375  
##  Median : 74.85            Median : 750  
##  Mean   : 78.85            Mean   : 750  
##  3rd Qu.: 88.11            3rd Qu.:1125  
##  Max.   :101.38            Max.   :1500
\end{verbatim}

\begin{Shaded}
\begin{Highlighting}[]
\NormalTok{donnees}
\end{Highlighting}
\end{Shaded}

\begin{verbatim}
##   densité mesure agrégation seuil 3 mesure agrégation seuil 4
## 1       2                    620.42                   1308.65
## 2      15                    234.82                   1039.02
## 3      33                    130.06                    809.17
##   mesure agrégation seuil 5 echelle
## 1                     74.85       0
## 2                    101.38     750
## 3                     60.31    1500
\end{verbatim}

On réalise un graphe afin de pouvoir comparer toutes ces données :

\begin{Shaded}
\begin{Highlighting}[]
\KeywordTok{plot}\NormalTok{(donnees}\OperatorTok{$}\NormalTok{densité,donnees}\OperatorTok{$}\NormalTok{echelle,}\DataTypeTok{type=}\StringTok{"b"}\NormalTok{,}\DataTypeTok{lwd=}\DecValTok{7}\NormalTok{,}\DataTypeTok{xlab=}\StringTok{"densité d'espaces vacants en %"}\NormalTok{,}\DataTypeTok{ylab=}\StringTok{"mesure d'agrégation(No)"}\NormalTok{,}\DataTypeTok{col=}\StringTok{"white"}\NormalTok{)}
\KeywordTok{lines}\NormalTok{(donnees}\OperatorTok{$}\NormalTok{densité,donnees}\OperatorTok{$}\StringTok{`}\DataTypeTok{mesure agrégation seuil 3}\StringTok{`}\NormalTok{,}\DataTypeTok{col=}\StringTok{"green"}\NormalTok{,}\DataTypeTok{lwd=}\DecValTok{4}\NormalTok{,}\DataTypeTok{type=}\StringTok{"b"}\NormalTok{)}
\KeywordTok{lines}\NormalTok{(donnees}\OperatorTok{$}\NormalTok{densité,donnees}\OperatorTok{$}\StringTok{`}\DataTypeTok{mesure agrégation seuil 4}\StringTok{`}\NormalTok{,}\DataTypeTok{col=}\StringTok{"red"}\NormalTok{,}\DataTypeTok{lwd=}\DecValTok{4}\NormalTok{,}\DataTypeTok{type=}\StringTok{"b"}\NormalTok{)}
\KeywordTok{lines}\NormalTok{(donnees}\OperatorTok{$}\NormalTok{densité,donnees}\OperatorTok{$}\StringTok{`}\DataTypeTok{mesure agrégation seuil 5}\StringTok{`}\NormalTok{,}\DataTypeTok{col=}\StringTok{"blue"}\NormalTok{,}\DataTypeTok{lwd=}\DecValTok{4}\NormalTok{,}\DataTypeTok{type=}\StringTok{"b"}\NormalTok{)}
\KeywordTok{legend}\NormalTok{(}\FloatTok{28.3}\NormalTok{,}\DecValTok{1580}\NormalTok{, }\DataTypeTok{legend=}\KeywordTok{c}\NormalTok{(}\StringTok{"Seuil 3"}\NormalTok{, }\StringTok{"Seuil 4"}\NormalTok{,}\StringTok{"Seuil 5"}\NormalTok{), }\DataTypeTok{col=}\KeywordTok{c}\NormalTok{(}\StringTok{"green"}\NormalTok{, }\StringTok{"red"}\NormalTok{,}\StringTok{"blue"}\NormalTok{), }\DataTypeTok{lty=}\DecValTok{1}\NormalTok{,}\DataTypeTok{lwd=}\DecValTok{4}\NormalTok{)}
\end{Highlighting}
\end{Shaded}

\includegraphics{projet1_files/figure-latex/unnamed-chunk-16-1.pdf}
SEUIL 3 : Le N0 diminue lorsque la DENSITE augmente: 620.96 pour 2\%,
236.92 pour 15\% et 129.31 pour 33\%. SEUIL 4 : Le N0 diminue lorsque la
DENSITE augmente : 1292.24 pour 2\%, 1046.18 pour 15\% et 95.25 pour
33\%. SEUIL 5 : Le N0 varie peu lorsque que la DENSITE augmente : 73.96
pour 2\%, 95.25 pour 15\% et 62.27 pour 33\%.

Dans toutes les simulations le N0 est plus faible pour le SEUIL 5, un
peu plus élevé pour le SEUIL 3 (même s'ils semblent se rapprocher du
SEUIL 5 plus la DENSITE augmente). Le N0 du SEUIL 4 est quant à lui bien
plus élevé que les autres.

\hypertarget{b-interpruxe9tations}{%
\section{B-Interprétations:}\label{b-interpruxe9tations}}

On voit bien que la densité et le seuil de tolérance on tous deux un
impact sur la ségrégation des populations différentes.

En augmentant les espaces vides globalement le N0 diminue, il y a moins
d'agents uniquement entourés d'agents du même type. Les agents ont moins
de chances d'être entourés de ``trop'' d'agents de types différents.
Mais il y a peu d'endroits où l'on retrouve assez d'agents du même type
pour en être totalement entouré.

Le seuil 5 est exigeant, il est compliqué de trouver des espaces vacants
qui satisferont le seuil pour les agents insatisfait. Cela explique le
faible N0. Peu d'agents auront 8 agents identiques autour d'eux. Il y a
du mélange entre les populations, car de toutes façon il est très
compliqué pour un agent de trouver un endroit qui lui conviendra. On
observe de nombreux petits ``quartiers''.

Le seuil 3 est peu exigeant, un agent ne demande ``moins'' de d'agents
du même type. Il y a par conséquent plus d'espaces vacants qui
conviendront aux agents insatisfait. Le mélange entre populations sera
tout de même un peu présent. On observe de nombreux quartiers légèrement
plus important que ceux du seuil 5.

Le seuil 4 est celui qui donnera le plus de ``ségrégation'' des
populations. Le seuil est assez élevé pour que les agents de types
différents cherchent à ``s'éviter'' mais il n'est pas trop élevé et
n'empêche pas de trouver des espaces vacants satisfaisant. On observe
moins de quartier, mais des quartier de populations bien plus important.
Il y a très peu de ``mélange''.

\hypertarget{iv-conclusion}{%
\subsection{IV CONCLUSION}\label{iv-conclusion}}

Pour conclure, on voit qu'en suivant le modèle de Schelling les données
changent selon le seuil et la densité, mais que dans tous les cas les
agents semblables ont tendance à se rassembler dans des zones entre eux
(``quartiers''). Les décisions individuelles de chacun des agents
entrâine une ségrégation spatiale collective.

On aurait aussi pu faire varier d'autres paramètres, telle que taille de
la matrice, la disposition de départ des agents (ici, on a choisi de les
disposer aléatoirement.) mais aussi les tailles des différentes
populations (on aurait pu mettre une population en infériorité
numérique.);

\end{document}
